
\documentclass[11pt]{article}

\usepackage{kotex}
\usepackage{amsmath}
\usepackage{amsthm}
\DeclareMathOperator*{\argmax}{arg\,max}
\DeclareMathOperator*{\argmin}{arg\,min}
\usepackage{amssymb}
\usepackage{amsbsy}
\usepackage{srcltx}
\usepackage{graphicx} 
\usepackage{amsfonts,latexsym}
\usepackage{blindtext}
\usepackage[utf8]{inputenc}
\usepackage[all]{xy}
\usepackage{indentfirst}


\voffset -1.5 true cm
\hoffset -1.5 true cm \textheight 8.8 true in \textwidth 6 true in
\linespread{1.2}
\newcommand{\be}{\begin{eqnarray*}}
\newcommand{\ee}{\end{eqnarray*}}

\title{ Assignment2}
\author{Taekang Eom}
\begin{document}

\maketitle
\thispagestyle{empty}

\section{Introdution to Code}
\begin{displaymath}
\xymatrix{ \textrm{message source} \ar[d] & & \textrm{receiver} \\
\textrm{source encoder} \ar[r] & \textrm{channel} \ar[r] & \textrm{source decoder}\ar[u]}
\end{displaymath}

 coding theory는  변조될 가능성이 있는 매체(보통 channel이라고 한다)를 통해 메세지를 보낼 때, 훼손된 메세지를 복구할 수 있는 방법에 대해 연구하는 학문이다. 메세지 $m$을 보낼 때, $m$보다 더 긴 메세지 $c$로 변환하는 encoding과정을 거쳐 channel을 통해 전송하고, decoding과정에서 channel을 거치면서 훼손된 메세지 $d$에서 $m$을 알아낸다.\\\\
{\bf Definition}
 $\Sigma$를 문자들의 집합이라고 하자. 이 때 $C \subset \Sigma^{n}$를 길이가 $n$인 Code라고 한다.\\\\
앞으로 $q=\vert\Sigma\vert$ 로 사용한다.\\\\
{\bf Definition}
$x=x_{1}x_{2}\dots x_{n}, y=y_{1}y_{2}\dots y_{n}\in\Sigma^{n}$일 때 Hamming distance $d$를
\begin{displaymath}
d(x,y) = d(x_{1},y_{1})+\dots+d(x_{n},y_{n})
\end{displaymath}
\begin{displaymath}
d(x,y)= \left\{ \begin{array}{ll}
1 & \textrm{if $x_{i}\neq y_{i}$}\\
0 & \textrm{if $x_{i}=y_{i}$}
\end{array} \right.
\end{displaymath}
 정의한다.\\\\
{\bf Properties}\\
1.$d(x,y)\geq 0$\\
2.$d(x,y)=0\iff x=y$\\
3.$d(x,y)=d(y,x) \quad\forall x,y\in\Sigma^{n}$\\
4.$d(x,y)+d(y,z)\geq d(x,z)\quad \forall x,y,z\in\Sigma^{n}$\\\\
{\bf Proof}
1,2,3은 자명.\\
4는 임의의 $i=1\sim n$에 대해 $d(x_{i},y_{i})+d(y_{i},z_{i})\geq d(x_{i},z_{i})$이므로 모든 $1\leq i\leq n$에 대해 더해주면 증명된다. ///\\\\
{\bf Definition}\\
1. $d(C) = \min\{d(x,y)\arrowvert x,y\in C, x\neq y\}$은 code $C$의 distance이다.\\
2. $x\in C, 0<d(x,y)\leq u \Rightarrow y\notin C$이면 C는 $u$-error-detecting이다. \\
3. $x\in C, d(x,y)\leq v \Rightarrow \argmin_{z\in C} d(y,z)=x$이면 C는 $v$-error-correcting이다.\\
4. $B_{d}(x,c)=\{y\in\Sigma^{n}\vert d(x,y)\leq c\}$를 $x$에서의 $c$-ball이라 한다.\\\\
{\bf Observation}
Code $C$에 대해 $d=d(C)$일때, C는 $d-1$-error-detecting이고, $\left\lfloor \frac{d-1}{2}\right\rfloor$-error-correcting이다.\\\\
\textbf{Example}
$C=\{000,111\}$은 길이가 3인 1-error-correcting code이다.\\\\
 이제 $C$의 길이와 distance가 정해져 있을 때 가능한 $C$의 최대 크기에 대해 알아보자. $C$의 최대 크기 중 가장 많이 알려진 2가지는 Hamming bound와 Singleton bound 이다\\\\\\
\textbf{Hamming Bound} 길이가 $n$인 $v$-error-correcting-code C에 대해 
\begin{displaymath}
\vert C\vert \leq \frac{q^{n}}{\sum_{i=0}^{v} \binom{n}{i}(q-1)^{i}}
\end{displaymath}
이 성립한다.\\\\
\textbf {Proof}
$C$는 $v$-error-correcting이므로 $x,y\in C, x\neq y$이면 $B_{d}(x,v)\cap B_{d}(y,v) = \emptyset$이다.//
 길이가 $n$인 문자열에서 $i$개의 문자가 틀린 경우, 틀릴 수 있는 위치가 $\binom{n}{i}$이고 각 위치마다 원래의 문자를 제외한 $q-1$개의 문자를 선택할 수 있으므로 경우의 수는 $\binom{n}{i}(q-1)^{i}$이다.\\
 따라서, 모든 $x\in C$에 대해 $\vert B_{d}(x,v)\vert = \sum_{i=0}^{v} \binom{n}{i}(q-1)^{i}$이 성립하고, $\Sigma^{n} \supset \bigcup_{x\in C}B_{d}(x,v)$이므로 $q^{n}\geq \left( \sum_{i=0}^{v} \binom{n}{i}(q-1)^{i} \right)\vert C\vert$ 이다. ///\\\\\\
 {\bf Singleton Bound} 길이가 $n$, distance가 $d$인 code $C$에 대해
 \begin{displaymath}
\vert C\vert \leq q^{n-d+1}
\end{displaymath}
이 성립한다.\\\\
{\bf Proof}
$x,y\in C, x\neq y$이면 $x,y$는 최소 $d$개의 문자가 다르기 때문에 맨 뒤의 $d-1$개의 문자를 지워도 서로 다르다. 따라서 $C$의 원소중 앞 $n-d+1$자리가 같은 원소는 없고, $\vert C\vert \leq q^{n-d+1}$임을 알 수 있다. ///



\section{Linear Codes and Hamming Codes}
\subsection{Linear Codes}

code의 정의에 따르면 code는 scalar가 finite field인 vector space위의 subspace로 잡을 수 있고, 이러한 code들을 linear code라 한다. code를 vector space로 생각하면 훨씬 더 간단해지기 때문에 실제로 사용되는 code들은 대부분 linear code이다.\\
{\bf Definition}\\
1. $C$가 $\mathbb{F}_{q}^{n}$의 subspace이면 $C$를 $\mathbb{F}_{q}$ 위에서 길이가 $n$인 linear code라 한다.\\
2. $\dim(C)$($C$를 $\mathbb{F}_{q}$ 위의 vectorspace로 볼 때의 dimension)을 linear code $C$의 dimension이라 한다.\\
3.길이가 $n, \dim(C)=k,d(C)=d$인 $\mathbb{F}_{q}$위의 linear code $C$를 $[n,k,d]_{q}$-code라 한다.

\subsection{Hamming Codes}

이 section에서는 1-error-correcting-code중 하나인 Hamming Code에 대해 소개할 것이다. Hamming code는 $[2^{r}-1,2^{r}-r-1,3]_{2}$-code 이다($r\geq2$).여기서는 $r=3$를 예시로 들어 Hamming code를 구성하는 방법을 설명 할 것이다.\\
길이가 7인 어떤 linear code $C \in \mathbb{F}_{2}^{7}$를 생각하자. $C$는 linear code이므로 $\mathbf{0} \in C$이다. $i$번째 자리만 1이고 나머지는 모두 0인 벡터를 $\mathbf{e}_{i}(1\leq i \leq 7)$  으로 두고, $H$를 아래와 같이 정의할 때, $\mathbf{H}\mathbf{e}_{i}$는 아래에서 부터 읽을 때, $i$의 2진수 표현임을 알 수 있다.
\begin{displaymath}
\mathbf{H} = \begin{pmatrix}1&0&1&0&1&0&1\\ 0&1&1&0&0&1&1\\0&0&0&1&1&1&1\end{pmatrix}
\end{displaymath}
마찬가지로 임의의 $\mathbf{v}\in C$에 대해 i번째 자리가 틀리면 $\mathbf{v}+\mathbf{e}_{i}$가 되고, 만약 $\mathbf{H}(\mathbf{v}+\mathbf{e}_{i})=\mathbf{H}\mathbf{e}_{i}$가 된다면 이 결과를 보고 $i$번째 자리가 틀렸음을 알 수 있다. 이 결과가 성립하는 최대의 $C$는 $\mathbf{H}$의 nullspace이다.  $C$를 조금 더 쉽게 구하기 쉽게 하기 위해 $\mathbf{H}$의 3,4번째 column을 바꿔 아래와 같이 만든다.
\begin{displaymath}
\mathbf{H} = \begin{pmatrix}1&0&0&1&1&0&1\\ 0&1&0&1&0&1&1\\0&0&1&0&1&1&1\end{pmatrix} = \left( \mathbf{I}_{r} \arrowvert  \mathbf{A}  \right)
\end{displaymath}
이 때, $\mathbf{G}$를 아래와 같이 두면 $\mathbf{H}\mathbf{G}^{T}=\mathbf{0}$ 이 됨을 알 수 있다.
\begin{displaymath}
\mathbf{G}= \left(  -\mathbf{A}^{T} \arrowvert  \mathbf{I}_{2^{r}-r-1}  \right) = \begin{pmatrix}1&1&0&1&0&0&0\\ 1&0&1&0&1&0&0\\0&1&1&0&0&1&0\\1&1&1&0&0&0&1\end{pmatrix}
\end{displaymath}
따라서,  $C$는  $\mathbf{G}$의 row space이다.

위의 내용에 기초해서 Hamming Code의 encoding과 decoding과정에 대해 설명할 것이다.\\
메세지 $\mathbf{m}\in \mathbb{F}_{2}^{2^{r}-r-1}$을 encoding하는 것은 $\mathbf{c}=\mathbf{m}\mathbf{G}$로 변환하면 된다.\\
channel을 통해 받은 $\mathbf{d}\in\mathbb{F}_{2}^{2^{r}-1}$을 decoding 하려면 먼저 어느 자리가 틀렸는지 알아야 한다. 이는 
$\mathbf{H}\mathbf{d}^{T}=\mathbf{H}\mathbf{e}_{i}$인 $i$를 알아내면 알 수 있다. $\mathbf{H}\mathbf{d}^{T}$을 아래에서 부터 읽었을 때, 이진수로 $l$가 되면 $i$는 다음과 같다. 
\begin{displaymath}
i= \left\{ \begin{array}{ll}
j +1& \textrm{if $l=2^{j}$}\\
l+r-j & \textrm{if $2^{j-1}<l<2^{j}$}
\end{array} \right.
\end{displaymath}
그러면 원래의 메세지 $\mathbf{m}$은 $\mathbf{m}\mathbf{G}=\mathbf{d}-\mathbf{e}_{i}$으로 encoding되었고, $\mathbf{m}$은 $\mathbf{d}-\mathbf{e}_{i}$의 뒷 $2^{r}-r-1$자리 이므로 $\mathbf{d}$로부터 $\mathbf{m}$으로 decoding을 했음을 알 수 있다.



\section{Reed-Solomon Codes}
이 section에서는 모든 finite field는 $\vert F\vert=p^{m}$($p$는 소수, $m\in\mathbb{N}$)을 만족해야 하고, 그런 $p,m$이 주어질 때, $\vert F\vert=p^{m}$인 field $F$도 항상 존재한다는 증명하지 않고 사용할 것이다. $\vert F\vert=p^{m}$인field는 $x^{p^{m}}-x\in\mathbb{F}_{p}[x]$를 $p^{m}$개의 해를 가지도록 확장한 field라고 생각하면 된다. $\vert F\vert=p^{m}$인 $F$를 $\mathbb{F}_{p^{m}}$로 표기하기로 한다.

Reed-Solomon Code는 $\mathbb{F}_{q}$위에서의 linear code이다.  그리고 Code 의 길이는 $n\leq q$(일반적으로 $n=q-1$을 가장 많이 사용), message의 길이는 $k\leq n$을 만족해야 한다. 이 때 이 Code의 distance는 $d=n-k+1$이 된다.  이 코드의 경우는 Singleton bound를 만족하기 때문에 효율이 좋은 Code(보내는 message대비 많은 오류를 교정할 수 있음)중 하나이며, 통신이나 저장매체에 많이 사용되었던 Code이기도 하다(이 때는 $\mathbb{F}_{2^{m}}$위에서의 Code를 사용하게 된다) . 이 Code를 Encoding하는 방법과 Decoding하는 방법에 대해 알아보자.
\subsection{Encoding}
Reed-Solomon Code를 encoding하는 방법에는 크게 2가지가 있다. 첫 번째 방법은 서로 다른 $b_{i}\in\mathbb{F}_{p^{m}}(1\leq i\leq n-k)$를 잡고,
\begin{displaymath}
g(x) = \prod_{i=1}^{n-k} (x-b_{i})
\end{displaymath}
로 둘 때, 메세지 $\mathbf{m}=(m_{1},m_{2},\dots,m_{k})\in\mathbb{F}_{q}^{k}$에 대해 $m(x) = \sum_{i = 1}^{k} m_{i}x^{i-1}$로 바꿔 $c(x) = g(x)m(x)$로 encoding하는 것이다. 이 방법은 직관적으로 이해하기는 쉽지만 decoding방법이 어려운 편이다. 따라서 encoding은 조금 비직관적이지만 decoding을 더 쉽게 할 수 있는 방법을 소개할 것이다.

두 번째 방법은 서로 다른 $a_{i}\in\mathbb{F}_{p^{m}}(1\leq i\leq n)$를 잡고, $\mathbf{m}=(m_{1},m_{2},\dots,m_{k})\in\mathbb{F}_{q}^{k}$을 아래와 같이 encoding하는 것이다($m(x)$는 첫 번째 방법에서 사용한 것과 같다).
\begin{displaymath}
\mathbf{c}=\left(m(a_{1}),m(a_{2}),\dots,m(a_{n})\right)\in\mathbb{F}_{q}^{n}
\end{displaymath}
실제로는 이와 같은 연산을 빠르게 하기 위해 FFT(Fast Fourier Transform)가 사용된다(Appnedix 참고).
\subsection{Decoding}
위에서 쓴 것처럼 여기에서는 두번째 encoding에 대한 decoding방법만 소개할 것이다. 이 알고리즘에서는 Euclidean Algorithm의 다항식 버전인
Extended Euclidean Algorithm(EEA)을 사용한다. $\deg r_{0}\geq\deg r_{1}$인 다항식 $r_{0},r_{1}\in\mathbb{F}_{q}[x]$에대한 EEA는 다음과 같다.\\
$i\in\mathbb{N}$에 대해 $r_{0},r_{1},\dots,r_{i}$까지 구해졌고, $r_{i}\neq0$ 이면 $r_{i+1}=r_{i-1}-q_{i}r_{i},\deg(r_{i+1})<\deg(r_{i})$를 만족하는 $q_{i},r_{i+1}$을 구한다. 이 때,  이 조건을 만족하는 $q_{i},r_{i+1}$는 유일하게 존재한다(증명은 Euclidean Algorithm에서와 거의 비슷하다).\\
 이 과정을 반복해 $r_{m+1}=0$을 얻으면 알고리즘을 종료한다(decoding할 때는 알고리즘을 완전히 수행하지 않고 중간에 끊을 것이다). 그 결과로 $\gcd(r_{0},r_{1})=r_{m}$을 얻는다.\\
 추가적으로, $u_{0}=1,u_{1}=0,v_{0}=0,v_{1}=1,u_{i+1}=u_{i-1}-q_{i}u_{i},v_{i+1}=v_{i-1}-q_{i}v_{i}\quad(1\leq i\leq m)$으로 두면 $r_{i} = u_{i}r_{0}+v_{i}r_{1}\quad(0\leq i\leq m+1)$, $\deg u_{i}+\deg r_{i-1}=\deg r_{1}, \deg v_{i}+\deg r_{i-1}=\deg r_{0}\quad(2\leq i\leq m+1)$임을 수학적 귀납법을 통해 얻을 수 있다.\\
 EEA를 이용한 decoding은 아래와 같이 이루어 진다.\\\\
{\large \textbf{Algorithm of Decoding Reed-Solomon Code}}\\
 Input: 변조된 메세지 $\mathbf{d}=(d_{1},d_{2},\dots,d_{n})\in\mathbb{F}_{q}^{n}$\\
 Output: 원래 전송했던 메세지  $\mathbf{m}=(m_{1},m_{2},\dots,m_{k})\in\mathbb{F}_{q}^{k}$ 또는 decoding 실패\\
\textbf{Step1} $\deg g_{1}(x)< n, g_{1}(a_{i})=d_{i}\quad(1\leq i\leq n)$인 $g_{1}\in\mathbb{F}_{q}[x]$을 찾는다. 이를 만족하는 $g_{1}$은 유일하게 존재한다. (Inverse FFT)\\
\textbf{Step2} $g_{0}(x) = \prod_{i=1}^{n} (x-a_{i}),g_{1}(x)$에 대해 $\deg (g(x))<\frac{1}{2}(n+k)$인 $g(x)$를 얻을 때까지 EEA를 적용해 아래와 같은 결과를 얻는다.
\begin{displaymath}
g(x) = u(x)g_{0}(x)+v(x)g_{1}(x)
\end{displaymath}
\textbf{Step3} $g$를 $v$로 나눈다. 몫과 나머지를 각각 $f_{1},r$이라 할 때, $r=0,\deg(f_{1})<k$이면 $f_{1}$이 output이고, 그 이외에는 decoding에 실패한다. \\\\
{\large \textbf{Proof of Algorithm}}\\
알고리즘의 정당성을 증명하기 위해 먼저 2개의 Lemma를 증명하겠다.\\
\textbf{Lemma1} EEA에서 $u_{m+1}=(-1)^{m+1}\frac{r_{1}}{r_{m}},v_{m+1}=(-1)^{m}\frac{r_{0}}{r_{m}}$\\
\textbf{Proof} 
\begin{displaymath}
\begin{pmatrix}u_{i}&v_{i}\\ u_{i+1}&v_{i+1}\end{pmatrix}=
\begin{pmatrix}0&1\\ 1&-q_{i}\end{pmatrix}\begin{pmatrix}u_{i-1}&v_{i-1}\\ u_{i}&v_{i}\end{pmatrix}
=\begin{pmatrix}0&1\\ 1&-q_{i}\end{pmatrix}\cdots\begin{pmatrix}0&1\\ 1&-q_{1}\end{pmatrix}\begin{pmatrix}u_{0}&v_{0}\\ u_{1}&v_{1}\end{pmatrix}
\end{displaymath}
\begin{displaymath}
\det\begin{pmatrix}u_{i}&v_{i}\\ u_{i+1}&v_{i+1}\end{pmatrix}=\prod_{j=1}^{i}\det \begin{pmatrix}0&1\\ 1&-q_{j}\end{pmatrix}=(-1)^{i}
\end{displaymath}
$\begin{pmatrix}r_{i}\\ r_{i+1}\end{pmatrix}=\begin{pmatrix}u_{i}&v_{i}\\ u_{i+1}&v_{i+1}\end{pmatrix}\begin{pmatrix}r_{0}\\ r_{1}\end{pmatrix}(0\leq i\leq m)$이므로 
\begin{displaymath}
\begin{pmatrix}r_{0}\\ r_{1}\end{pmatrix}=\begin{pmatrix}u_{i}&v_{i}\\ u_{i+1}&v_{i+1}\end{pmatrix}^{-1}\begin{pmatrix}r_{i}\\ r_{i+1}\end{pmatrix}
=(-1)^{m}\begin{pmatrix}v_{i+1}&-v_{i}\\ -u_{i+1}&u_{i}\end{pmatrix}\begin{pmatrix}r_{i}\\ r_{i+1}\end{pmatrix}
\end{displaymath}
$i=m$이면 $r_{m+1}=0$이므로 $r_{0}=(-1)^{m}v_{m+1}r_{m}, r_1=(-1)^{m+1}u_{m+1}r_{m}$을 얻는다. ///\\\\
\textbf{Lemma2} $\deg r_{0}\leq t, \deg \epsilon_{i} \leq l\quad (i=0,1),\deg w_{0}\geq d_{0}>l+t,\gcd(r_{0},r_{1})=1$을 만족하는 $d_{0},l,t\in\mathbb{Z},w_{0},r_{i},\epsilon_{i}\in \mathbb{F}_{q}[x]$가 존재한다고 하자. $g_{0}=w_{0}r_{0}+\epsilon_{0},g_{1}=w_{0}r_{1}+\epsilon_{1}$에 대해 EEA를 적용하고, $\deg g<d_{0}$이면 멈춘다. EEA의 결과로 $g=ug_{0}+vg_{1}$를 얻을 때, $0$이 아닌 $\alpha\in\mathbb{F}_{q}\backslash\{0\}$가 존재해 $u(x)=-\alpha r_{1}(x),v(x)=\alpha r_{0}(x)$가 된다.\\
\textbf{Proof} 먼저 $r_{0},r_{1}$에 대해 EEA를 적용해 $r_{m+1}=0$이 되었다고 가정하자. 이 때, $g_{0},g_{1}$에 대해 EEA를 적용할 때, , $r_{0},r_{1}$에 대해 EEA를 적용할 때와 같은 $q_{i},u_{i},v_{i}$를 얻고, 위의 방법대로 멈추면 정확히 $k=m+1$이 됨을 증명 할 것이다.\\
$g_{i}=u_{i}g_{0}+v_{i}g_{1}\quad(2\leq i\leq m+1)$로 두면 $g_{i+1}=g_{i-1}-q_{i}g_{i}\quad(1\leq i\leq m)$ 를 얻는다.
\begin{displaymath}
g_{i}=u_{i}(w_{0}r_{0}+\epsilon_{0})+v_{i}(w_{0}r_{1}+\epsilon_{1})=w_{0}(u_{i}r_{0}+v_{i}r_{1})+(u_{i}\epsilon_{1}+v_{i}\epsilon_{1})=w_{0}r_{i}+(u_{i}\epsilon_{1}+v_{i}\epsilon_{1})
\end{displaymath}
이고,$\deg(u_{i}\epsilon_{1}+v_{i}\epsilon_{1})\leq l+t\leq d_{0}$ ($\deg u_{i},\deg v_{i}\leq \deg{r_0}$)이므로, 
\begin{displaymath}
\deg g_{i} = \deg g_{0}+\deg r_{i}\geq \deg w_{0} \geq d_{0}(0\leq i\leq m),
\end{displaymath}
\begin{displaymath}
\deg g_{m+1} = \deg (u_{m+1}\epsilon_{1}+v_{m+1}\epsilon_{1})\leq l+t<d_{0}
\end{displaymath}
을 얻는다. 또한, EEA의 정의에 따라, $r_{1},r_{2},\dots,r_{m}$의 degree는 감소하고, 따라서 $g_{1},g_{2},\dots,g_{m+1}$의 degree도 감소해야 함을 알 수 있다. 따라서 $g_{0},g_{1}$에 대해 EEA를 적용할 때, , $r_{0},r_{1}$에 대해 EEA를 적용할 때와 같은 $q_{i},u_{i},v_{i}$를 얻고, 위의 방법대로 멈추면 정확히 $k=m+1$이 되고, $g=g_{m+1}$임을 증명하였다.\\
또한, $\gcd(r_{0},r_{1})=1$이므로 $r_{m}\in\mathbb{F}_{q}\backslash\{0\}$이다. $\alpha = \frac{(-1)^{m}}{r_{m}}$으로 두면 Lemma1에 의해   $g_{m+1}g_{0}+v_{m+1}g_{1}$에서 $g=-\alpha r_{1}g_{0}+\alpha r_{0}g_{1}$을 얻는다. ///\\\\
\textbf{Proof of the Algorithm}\\
변조된 메세지 $\mathbf{d}=(d_{1},d_{2},\dots,d_{n})$과 $f(x)$를 통해 encoding했던 $\mathbf{c}=(c_{1},c_{2},\dots,c_{n})$ 와의 hamming distance가 $t\leq \frac{d-1}{2}$를 만족한다고 가정하자. $w,w_{0}$을
\begin{displaymath}
w(x)=\prod_{1\leq i\leq n, \\ c_{i}\neq d_{i}}(x-a_{i}), g_{0}(x)=w_{0}(x)w(x)
\end{displaymath}
로 정의하면 $\deg w =t$가 된다. 그리고 inverse FFT를 통해 $\deg\bar{w}<t, c_{i}\neq d_{i},1\leq i\leq n$를 만족하는 모든 $i$에 대해 $\bar{w}(a_{i})=\frac{d_{i}-c_{i}}{w_{0}(a_i)}$가 성립하는 유일한 $\bar{w}\in\mathbb{F}_{q}[x]$를 구할 수 있다.\\
또한 모든 $1\leq i\leq n$에 대해 $g_{1}(a_{i})=w_{0}(a_{i})\bar{w}(a_{i})+f(a_{i})=c_{i}$이고, $\deg f<k$이므로 $g_{1}=w_{0}\bar{w}+f$가 성립한다.\\
$d_{0}=\frac{n+k}{2}$로 둘 때, $\gcd(w,\bar{w})=1$이고, $\deg w_{0}=n-t\geq d_{0}\geq k-1+t\geq \deg f+\deg w$이므로 $g_{0},g_{1}$에 대해 EEA를 적용하면 Lemma2에 의해 $g(x)=u(x)g_{0}(x)+v(x)g_{1}(x)=-\alpha\bar{w}(x)g_{0}(x)+\alpha w(x)g_{1}(x)$를 만족하는 $\alpha\in\mathbb{F}_{q}\backslash\{0\}$가 존재한다.\\$g=-\alpha\bar{w}g_{0}+\alpha wg_{1}=-\alpha\bar{w}g_{0}+\alpha w(w_{0}\bar{w}+f)=\alpha wf=vf$이므로 $g$를 $v$로 나눈 나머지는 0이 되어야 하고, 이 알고리즘에서 얻은 결과는 $f$로 원하는 결과를 얻었다.\\
반대로, 이 알고리즘을 통해 $f_{1}(x)(\deg f_{1}<k)$를 얻었다고 가정하자.  Step2,Step3에서 얻은 식에 의해 $u(x)g_{0}(x)=v(x)(f_{1}(x)-g_{1}(x))$이고, 따라서 $v(a_{i})(f_{1}(a_{i})-g_{1}(a_{i}))=0(1\leq i\leq n)$이다. EEA를 적용한 결과가 $g(x)=g_{m+1}(x)$라고 하면 $\deg g_{m}\geq \frac{n+k}{2}$이므로, $\deg v = \deg g_{0}-\deg g_{m}\leq n-\frac{n+k}{2}=\frac {d-1}{2}$를 얻는다. 따라서 $n-\frac{d-1}{2}$개 이상의 $i$에 대해 $f_{1}(a_{i})=g_{1}(a_{i})$이 성립하고, $f_{1}$으로 encoding된 메세지와 channel을 통해 받은 메세지의 hamming distance가 $\frac{d-1}{2}$이하임을 알 수 있다. 이 사실을 통해 알고리즘이 제대로 동작함을 알 수 있다. ///
\section{Application for Group Testing Problem}
 Group Testing Problem은 2차 세계대전 때, 미국에서 군에 지원하는 사람들을 대상으로 매독검사를 효율적으로 하기 위해 고안된 문제이다. 이 때, 매독검사의 비용이 비쌌고 감염된 사람의 비율도 적었기 때문에 정부에서는 최소한의 검사를 통해 감염된 사람을 추려내고 싶어했고, 이 문제를 당시 통계학자이자 미국 연방 정부의 연구원이었던 Robert Dorfman이 1943년 부분적인 해법을 제시하면서 수학자들에게 알려지게 되었다. 이 문제를 조금 더 정확하게 서술하면 다음과 같다.
 
$n$명의 사람이 있고, 그 중 $k\leq n$명 이하의 사람이 감염되어 있다는 사실을 알고 있다고 가정하자. 검사 $T$번 한다고 가정할 때, $A_{i}\subset \{1,2,\dots,T\}\quad(1\leq i\leq n)$이 $i$번째 사람이 받은 검사들의 집합이라 하고, $\mathbb{A}=\{A_{1},\dots,A_{n}\}$으로 두자. 만약 어떤 검사가 음성 반응이 나오면 그 검사를 받은 사람들은 모두 감염되지 않았다고 확신할 수 있다. 따라서 $\mathbb{A}$에서 감염된 $k$명과 대응되는 $A_{i}$들의 합집합을 구한 결과가 $A$라 할 때, $A_{i}\nsubseteq A$이면 $i$번 째 사람은 감염된 사람들이 아무도 받지 않은 검사를 받았다는 것이고, $i$번 째 사람이 감염되지 않았음을 확신할 수 있다. 따라서$\mathbb{A}$에서 임의의 집합 $k$개의 합집합을 구했을 때, 나머지 집합에서 그 합집합에 속하지 않은 원소가 항상 나오게 되면 그 검사로 감염된 사람들을 정확히 골라낼 수 있다. 이 때 $T$의 값을 최소화 하는 것이 이 문제의 목표이다.\\
Reed-Solomon Code를 이용하면 $T=O\left(\left(k\frac{\log n}{\log k} \right)^{2}\right)$가 되도록 $T$를 잡을 수 있다. 증명은 아래와 같다.\\
\textbf{Proof}이 증명에서는 Reed-Solomon Code인 $[q,m,q-m+1]_{q}$ Code를 사용할 것이고 이 Code를 $C$라 하자.  $\vert\mathbb{F}_{q}\vert=q$이므로 bijective $f:\mathbb{F}_{q}\to \{1,\dots,q\}$가 존재한다.  이 때, $c=(c_{1},\dots,c_{q})\in C$에 대해 $A_{c}=\{q(i-1)+f(c_{i})\lvert1\leq i\leq q\}\subset\{1,\dots,q^{2}\}$로 정의하고, $\mathbb{A}=\{A_{c}\vert c\in C\}$라 두자.  이 때, 모든 $c\in C$에 대해 $\vert A_{c}\vert=q$이고, 서로 다른 $c_{1},c_{2}\in C$에 대해  $\vert A_{c_1}\cap A_{c_2}\vert=q-d(c_{1},c_{2})\leq q-(q-m+1)=m-1$이다.\\
$k=\lfloor\frac{q-1}{m-1}\rfloor$로 두면 서로 다른 $c,c_{1},\dots,c_{k}\in C$에 대해
\begin{displaymath}
\vert A_{c}\cap (A_{c_{1}}\cup\dots\cup A_{c_{k}} )\vert=\arrowvert\bigcup_{i=1}^{k}(A_{c}\cap A_{c_{i}}) \arrowvert\leq\sum_{i=1}^{k}\vert A_{c}\cup A_{c_{i}}\vert\leq k(m-1)\leq q-1
\end{displaymath}
이 성립하고,
\begin{displaymath}
 A_{c}\nsubseteq A_{c_{1}}\cup\dots\cup A_{c_{k}} 
\end{displaymath}
이 된다. 따라서, 만약, $k$명 이하가 감염되었다는 사실을 알면, $n=q^{m}$명의 사람에 대해 $T=q^{2}$번의 test를 통해 감염된 사람들을 모두 골라낼 수 있다는 사실을 알 수 있다.\\
$n,k$가 주어진 경우 $2k\frac{\log n}{\log k}\leq q \leq4k\frac{\log n}{\log k}$인 소수 $q$를 잡을 수 있다(by Bertrand's postulate). $m=\left\lceil\frac{\log n}{\log q}\right\rceil$으로 잡으면, 
\begin{displaymath}
\left\lfloor\frac{q-1}{m-1}\right\rfloor \geq \left\lfloor\frac{2k\frac{\log n}{\log k}-1}{\frac{\log n}{\log q}+1-1}\right\rfloor= \left\lfloor2k\frac{\log q}{\log k}-\frac{\log q}{\log n}\right\rfloor\geq k
\end{displaymath}
이고, $T=O\left(\left(k\frac{\log n}{\log k} \right)^{2}\right)$를 만족함을 알 수 있다. ///

\section{Fast Fourier Transform(FFT)}
이 section에서는 Reed-Solomon Code의 encoding과 decoding을 빠르게 할 수 있게 하는 Fast Fourier Transform(FFT)에 대해 다룰 것이다. 이를 이해하기 위해서는 finite field의 성질들과 추상적인 vector space에 대한 이해가 필요하므로 현대대수학을 수강하지 않은 경우 이 section을 무시해도 좋다. 이 section에서는 편의상 $n=p^{m}-1$로 둔다.

$\mathbb{F}_{p^{m}}$위에서 $c(x)\in\mathbb{F}_{p^{m}}[x]$ Fourier Transform $C$는 $C_{k}=c(\alpha^{k})\quad(0\leq k\leq n-1)$ 로 정의한다($\alpha$는 $\mathbb{F}_{p^{m}}$의 primitive element). $c(x)=\sum_{k=0}^{n-1}c_{k}x^{k}$로 나타낼 수 있을 때, Fourier Transform을 행렬로 나타내면 다음과 같다.\\
\begin{displaymath}
 \begin{pmatrix}C_{0}&C_{1}&\cdots&C_{n-1}\end{pmatrix}=
\begin{pmatrix}c_{0}&c_{1}&\cdots&c_{n-1}\end{pmatrix}
 \begin{pmatrix}1&1&\cdots&1\\1&\alpha&\cdots&\alpha^{n-1}\\ \vdots&\vdots&\ddots&\vdots\\1&\alpha^{n-1}&\cdots&\alpha^{(n-1)(n-1)}\end{pmatrix}
\end{displaymath}
반대로, $C_{k}\in\mathbb{F}_{p^{m}}(0\leq k\leq n-1)$이 주어져 있을 때, $C_{k}=c(\alpha^{k})$를 만족하는 $c(x)=\sum_{k=0}^{n-1}c_{k}x^{k}$를 구하는 것을 Inverse Fourier Transform이라 한다.
\begin{displaymath}
 \begin{pmatrix}1&1&\cdots&1\\1&\alpha&\cdots&\alpha^{n-1}\\ \vdots&\vdots&\ddots&\vdots\\1&\alpha^{n-1}&\cdots&\alpha^{(n-1)(n-1)}\end{pmatrix}
 \begin{pmatrix}1&1&\cdots&1\\1&\alpha^{-1}&\cdots&\alpha^{-(n-1)}\\ \vdots&\vdots&\ddots&\vdots\\1&\alpha^{-(n-1)}&\cdots&\alpha^{-(n-1)(n-1)}\end{pmatrix}
 =n\mathbf{I}_{n}
\end{displaymath}
가 성립하고, $n\equiv-1\phantom{a}(\textrm{mod } p)$이기 때문에, Inverse Fourier Transform은 행렬로 표현하면 다음과 같다.
\begin{displaymath}
\begin{pmatrix}c_{0}&c_{1}&\cdots&c_{n-1}\end{pmatrix}=
 -\begin{pmatrix}C_{0}&C_{1}&\cdots&C_{n-1}\end{pmatrix}
 \begin{pmatrix}1&1&\cdots&1\\1&\alpha^{-1}&\cdots&\alpha^{-(n-1)}\\ \vdots&\vdots&\ddots&\vdots\\1&\alpha^{-(n-1)}&\cdots&\alpha^{-(n-1)(n-1)}\end{pmatrix}
\end{displaymath}
따라서, Inverse Fourier Transform도 Fourier Transform과 비슷한 방법으로 할 수 있다.

Fast Fourier Transform은 $c(x)=\sum_{k=0}^{n-1}c_{k}x^{k}\in\mathbb{F}_{p^{m}}[x]$이 주어져 있을 때,  $(c(\alpha^{0}),\dots,c(\alpha^{n-1}))$를 빠르게 계산하는 것이 목적이다. $\alpha^{k}(0\leq k\leq n-1)$의 값을 모두 알고 있다고 가정할 때, 정의에 따라 계산하면 $O(n^2)$번의 덧셈과 곱셈을 통해 계산할 수 있다.   FFT를 통해 $O(n(\log n)^2)$번의 덧셈과 곱셈으로 $(c(\alpha^{0}),\dots,c(\alpha^{n-1}))$를 얻을 수 있다. 이 알고리즘을 간단히 설명하면 $C_{\alpha^{j}}$를 구하기 위해 $c(x)$를 $x-\alpha^{j}$로 나누는 방법을 사용했고, 빠르게 계산하기 위해 적절한 다항식을 골라서 여러번 나눗셈을 해준 것이다.

먼저 FFT를 수행하는데 필요한 정리를 설명한 후, FFT 알고리즘에 대해 설명할 것이다. \\
\textbf{Theorem} $\mathbb{F}_{p^{m}}$의 원소들을 $\beta_{0,0},\beta_{0,1},\dots,\beta_{0,p^{m}-1}$로 재배열해서 다음과 같은 성질을 만족하도록 할 수 있다.
\begin{displaymath}
q_{0,j}(x)=x-\beta_{0,j}\phantom{a}(0\leq j<p^{m})
\end{displaymath}
\begin{displaymath}
q_{i,j}(x)=\prod_{k=0}^{p-1}q_{i-1,jp+k}(x),Q_{i,j}=\prod_{k=0}^{p^{i}-1}\beta_{0,jp^{i}+k}\phantom{a}(1\leq i\leq m,0\leq j< p^{m-i})
\end{displaymath}
로 정의하면 $q_{i,0}$은
\begin{displaymath}
q_{i,0}=\sum_{k=0}^{i}c_{i,k}x^{p^{k}}
\end{displaymath}
와 같은 꼴이 되고, $q_{i,j}(x)=q_{i,0}(x)-Q_{i,j}$를 만족한다.\\\\
{\large\textbf{Algorithm of FFT}}\\
Input: $c(x)\in\mathbb{F}_{p^{m}}[x](\deg c <n)$\\
Output: $(c(\alpha^{0}),\dots,c(\alpha^{n-1}))$\\
\textbf{Step1} $r_{m,0}=c(x),i=m$\.\\
\textbf{Step2} $0\leq j< p^{m-i}$에 대해 $r_{i-1,jp+k}(x)=r_{i,j}(x)\mod(q_{i-1,jp+k}(x))$\\
\textbf{Step3} $i=i-1$\\
\textbf{Step4} $i>0$이면 Step2, $i=0$이면 Step5\\
\textbf{Step5} $0\leq j< p^{m}-1$에 대해 $\beta_{0,k}=\alpha^{j}$인 $j$를 찾고, $C_{j}=r_{0,k}$\\
\textbf{Step6} return $(C_{0},\dots,C_{p^{m}-2})$\\\\
이 알고리즘의 Step2에서 각각의 나눗셈에 대해 최대 $(p^{i}-p^{i-1})(i+1)$번의 덧셈과 곱셈이 각각 이루어지고 $p^{m-i+1}$번의 나눗셈을 하므로 각 $i$에서 $p^{m-i+1}(p^{i}-p^{i-1})(i+1)=(p^{m+1}-p^{m})(i+1)$번의 연산이 이루어진다. 모든 $1\leq i\leq m$에 대해 더해주면 $n=p^{m}-1,m=\log_{p}p^{m}$ 이므로 이 알고리즘에서 $O(n(\log n)^{2})$번의 연산이 이루어짐을 알 수 있다.\\\\
Theorem을 증명하기 전 증명에 필요한 Lemma를 증명하겠다.\\
\textbf{Lemma} $\mathbb{F}_{p^{m}}$을 $\mathbb{F}_{p}$-vector space로 볼 때, $V$를 $\mathbb{F}_{p^{m}}$의 subspace라 하자. 이 때,$\dim V=d$이면, 
\begin{displaymath}
f_{V}(x):=\prod_{v\in V}(x-v)=\sum_{k=0}^{d}c_{k}x^{p^{k}}
\end{displaymath}
인 $c_{0},\dots,c_{d}\in \mathbb{F}_{p^{m}}$이 존재한다.\\
\textbf{Proof} $0\leq i\leq d$에 대해 $f_{i}(x):=x^{p^{i}}$는 $\mathbb{F}_{p}$-linear function이고, $\{f\vert f:V\to\mathbb{F}_{p^{m}}\textrm{은 }\mathbb{F}_{p}\textrm{-linear}\}$은 dimension이 $d$인 $\mathbb{F}_{p^{m}}$-vector space이다($\{\gamma_{1},\dots,\gamma_{d}\}$가 $V$의 basis일 때, $f(\gamma_{1}),\dots,f(\gamma_{d})$가 정해지면 $f$가 유일하게 정해지기 때문에 dimension이 $d$인 $\mathbb{F}_{p^{m}}$-vector space로 생각할 수 있다).\\
따라서, $f_{0},\dots,f_{d}$는 linearly dependent하고, 모든 $x\in V$에 대해 $g_{V}(x)=0$을 만족하는 $g_{V}(x)=\sum_{k=0}^{d}a_{k}x^{p^{k}}\in\mathbb{F}_{p^{m}}[x], g_{V}\neq 0 $가 존재한다.$f_{V}\vert g_{V}, \deg g_{V}\leq \deg f_{V}=p^{d}$이므로 $\deg g_{V}=\deg f_{V},a_{d}\neq 0$이다. 따라서
\begin{displaymath}
f_{V}(x)=\sum_{k=0}^{d}\frac{a_{k}}{a_{d}}x^{p^{k}}
\end{displaymath}
가 성립한다. ///\\
\textbf{Proof of the Theorem}\\
$\mathbb{F}_{p^{m}}$을 $\mathbb{F}_{p}$-vector space로 볼 때, $\mathbb{F}_{p^{m}}$의 basis를 $\{\gamma_{0},\dots,\gamma_{m-1}\}$라 하자. $k\in\mathbb{Z},0\leq k<p^m$이 $k=\sum_{i=0}^{m-1}a_{i}p^{i}(0\leq a_{i}<q)$로 표현될 때, $\beta_{0,k}=a_{0}\gamma_{0}+\dots+a_{m-1}\gamma_{m-1}$로 정의하자.
\begin{displaymath}
U_{j}^{i}=\{\beta_{0,k}\vert jp^{i}\leq k< (j+1)p^{i}\}\quad(1\leq i\leq m,0\leq j< p^{m-i})
\end{displaymath}
로 정의하면 $U_{0}^{i}(1\leq i\leq m)$는 $U_{0}^{i}\supset U_{0}^{i-1}$를 만족하는$\mathbb{F}_{p^{m}}$의  subspace 이고,\\
 $U_{j}^{i}=\{\beta+\beta_{0,jp^{i}}\vert \beta\in U_{0}^{i}\}$이다. 따라서
 \begin{displaymath}
U_{j}^{i}=\bigcup_{k=0}^{p-1}U_{jp+k}^{i-1}\quad(1\leq i\leq m,0\leq j< p^{m-i})
\end{displaymath}
로 분할할 수 있고, 이를 이용하면 수학적 귀납법으로
\begin{displaymath}
q_{i,j}(x)=\prod_{\beta\in U_{j}^{i}}(x-\beta)
\end{displaymath}
임을 보일 수 있다.\\
Lemma에 의해 $q_{i,0}$은 $q_{i,0}(x)=\sum_{k=0}^{i}c_{i,k}x^{p^{k}}$와 같은 꼴로 나타낼 수 있다. 또한,
\begin{displaymath}
q_{i,j}(x)=\prod_{\beta\in U_{j}^{i}}(x-\beta)=\prod_{\beta\in U_{0}^{i}}(x-\beta-\beta_{0,jp^{i}})=q_{i,0}(x-\beta_{0,jp^{i}})
\end{displaymath}
\begin{displaymath}
=\sum_{k=0}^{i}c_{i,k}(x-\beta_{0,jp^{i}})^{p^{k}}=\sum_{k=0}^{i}c_{i,k}x^{p^{k}}+\sum_{k=0}^{i}c_{i,k}(-\beta_{0,jp^{i}})^{p^{k}}
\end{displaymath}
\begin{displaymath}
=q_{i,0}(x)+q_{i,0}(-\beta_{0,jp^{i}})=q_{i,0}(x)+\prod_{\beta\in U_{0}^{i}}(-\beta_{0,jp^{i}}-\beta)=q_{i,0}(x)+\prod_{\beta\in U_{j}^{i}}(-\beta)
\end{displaymath}
\begin{displaymath}
=q_{i,0}(x)+\prod_{\beta\in U_{j}^{i}}(-\beta)=q_{i,0}(x)+(-1)^{p^{i}}Q_{i,j}=q_{i,0}(x)-Q_{i,j}
\end{displaymath}
이다.  ///
\end{document}